\documentclass[a4paper,12pt]{article}

% Packages for headers, footers, and general formatting
\usepackage[utf8]{inputenc} % Encoding
\usepackage{fancyhdr}       % Headers and footers
\usepackage{enumitem}       % For advanced lists
\usepackage[luatex]{hyperref}       % For clickable URLs, citations etc
\usepackage[usenames,dvipsnames]{color}%has to be called before tikz
\usepackage{tikz}
\usepackage{wrapfig}
\usepackage{adjustbox}
%\usepackage{libertinus}
%\usepackage{setspace}
%\setstretch{1.25}
\usetikzlibrary{graphs,graphdrawing}
\usegdlibrary{trees}
\usepackage{color}
\usepackage[english,slovene]{babel} %language support
\usepackage{listings}
\usepackage[mode=buildnew]{standalone}
% Header and Footer Setup
\pagestyle{fancy}
\fancyhf{} % Clear all header and footer fields
%\fancyhead[R]{Your Header on the Left} 
%lstlistings popravek levega roba
%v glavi izmerim širino dveh majhnih številk in izračunam rob
\newlength{\MaxSizeOfLineNumbers}%
\settowidth{\MaxSizeOfLineNumbers}{\tiny 99}% Adjust to maximum number of lines
\addtolength{\MaxSizeOfLineNumbers}{2.5ex}%

\definecolor{lightgreen}{HTML}{CCFF99}

\fancyhead[C]{Optimizacija zaznave trkov v 2D}
\fancyfoot[C]{\thepage}
\setlength{\textwidth}{16cm}
\setlength{\oddsidemargin}{0cm}
\setlength{\headwidth}{\textwidth}

\lstset{
    numbers=left,
    numberstyle=\tiny,
    basicstyle=\ttfamily\small,
    keywordstyle=\color{violet},
    identifierstyle=\color{teal},
    extendedchars=true,
    xleftmargin=\MaxSizeOfLineNumbers,
    texcl=true,
    captionpos=b,
}

% \linespread{0.5}
% Begin Document
\begin{document}
\makeatletter
\addto\captionsslovene{
  % handle hyperref's autoref 
  \renewcommand\equationautorefname{enačba}%
  \renewcommand\footnoteautorefname{opomba}%
  \renewcommand\itemautorefname{alineja}%
  \renewcommand\figureautorefname{slika}%
  \renewcommand\tableautorefname{tabela}%
  \renewcommand\partautorefname{del}%
  \renewcommand\appendixautorefname{priloga}%
  \renewcommand\chapterautorefname{poglavje}%
  \renewcommand\sectionautorefname{razdelek}%
  \renewcommand\subsectionautorefname{podrazdelek}%
  \renewcommand\subsubsectionautorefname{podpodrazdelek-section}%
  \renewcommand\paragraphautorefname{odstavek}%
  \renewcommand\subparagraphautorefname{del odstavka}%
  \renewcommand\FancyVerbLineautorefname{vrstica}%
  \renewcommand\theoremautorefname{teorem}%
  \renewcommand\pageautorefname{stran}%
  % handle algorithm2e
  % \renewcommand{\listalgorithmcfname}{Seznam algoritmov}%
  % \renewcommand{\algorithmcfname}{Algoritem}%
  % \renewcommand{\algorithmautorefname}{\algorithmcfname}%
  % \renewcommand{\algorithmcflinename}{Vrstica}%
  % \renewcommand{\algocf@typo}{}%
  % \renewcommand{\@algocf@procname}{Procedura}%
  % \renewcommand{\@algocf@funcname}{Funkcija}%
  % \renewcommand{\procedureautorefname}{\@algocf@procname}%
  % \renewcommand{\functionautorefname}{\@algocf@funcname}%
  % \renewcommand{\algocf@languagechoosen}{slovene}%
  % handle lstlistings
  \renewcommand{\lstlistingname}{Izpis}%
}
\makeatother
 %slovene captions for hyperref and algorithm2e
% Title Page
\begin{titlepage}
    \title{\Huge Optimizacija zaznave trkov v 2D}
    \author{Klemen Javoršek}
    \date{Ljubljana, 2024/2025}
    \maketitle
    \renewcommand{\headrulewidth}{0cm}
    \fancyhf{}
    \fancyfoot[L]{Mentor: Klemen Bajec}
    \fancyfoot[R]{Gimnazija Vič}
    \thispagestyle{fancy}
    %\setcounter{page}{1}  
\end{titlepage}
\selectlanguage{slovene}

% Table of Contents
\newpage
\quad
\thispagestyle{empty}
\newpage
\tableofcontents
\newpage
\section{Uvod}

\textcolor{red}{Uvod so moji zapiski, ki jih bom kmalu spremenil v smiselno obliko.
Manjka mi še nekaj literature, proti koncu pa bom tudi popravil, kako se prelamlja
vsebina, da bo lepše.}

Kaj je zaznava trkov?

Zaznava trkov je pomembna na več različnih področjih:
\begin{itemize}
    \item Programska oprema za računalniško podprt razvoj (orodja CAD)
    \item Robotika
    \item Fizikalne simulacije
    \item Videoigre
\end{itemize}
Problemi:
\begin{itemize}
    \item Iskanje trkov je samo po sebi kvadratno kompleksno.
    \item Zato je večina postopkov pospeševanja zaznave trkov samo filtriranje parov objektov, ki se ne morejo zaleteti.
    \item Samo filtriranje je odvisno od pristopa, a ponavadi poskuša zagotoviti, da se časovno potratno izvajanje zaznave trkov zgodi na čim manjšem številu objektov.
    \item Za filtriranje so velikokrat potrebne podatkovne strukture, ki sistem, v katerem iščemo trke, razdelijo na prostorske regije, kar omogoči hitro zmanjševanje števila objektov, ki jih moramo obravnavati.
\end{itemize}


Navajeni smo, da učinkovitost merimo samo z asimptotičnimi vrednostmi ($1000 \cdot n^2$ je enako
kompleksno kot $10 \cdot n^2$), saj v svetu algoritmov in dokazovanja kompleksnosti ni smiselnega načina,
s katerim bi lahko določili te konstantne vrednosti.
A kljub temu so te konstantne vrednosti zelo pomembne za aplikacije algoritmov v resničnem svetu,
saj lahko (pri nizkem številu objektov) kljub svoji logaritemski kompleksnosti iskalni algoritem
s konstantnim faktorjem 850 močno zaostaja za hitrim, a kvadratno kompleksnim algoritmom za zaznavo
trkov s konstantnim faktorjem 16. Zato je pazljivo eksperimentiranje in določanje mejnih vrednosti ključnega
pomena za dobro delovanje programov za zaznavo trkov.

Primeri podatkovnih struktur:
\begin{itemize}
    \item Hierarhija omejujočih volumnov (R* tree)
    \item Quadtree -- drevo, ki rekurzivno razdeljuje 2D prostor na 4 dele
    \item Octree -- verzija drevesa quadtree, ki počne enako s 3D prostorom (in ga razdeli na 8 delov)
    \item k-dimenzionalno drevo -- drevo, ki rekurzivno razdeljuje prostor po izmenjujočih se dimenzijah
    \item BSP drevo -- Dvojiško drevo, ki obravnava >1D prostor.
\end{itemize}


Vse te podatkovne strukture organizirajo določena telesa, objekte. Ti objekti so lahko definirani na veliko različnih načinov. Ponavadi se ne organizira direktno objektov, ampak njihove omejujoče volumne. Ključna lastnost omejujočega volumna je:
Če trčita objekta, vedno trčita tudi njuna omejujoča volumna.
Na podlagi te lastnosti lahko zaključimo, da se začetni del zaznave trkov lahko izvaja samo z omejujočimi volumni in bomo s tem zaznali tudi vse trke objektov.
Ko imamo seznam trkov omejujočih volumnov, lahko začnemo s časovno bolj potratnim zaznavanjem trkov med objekti.



Implementirali smo fizikalni simulator v 2D prostoru, ki zaznava trke med omejujočimi volumni
v obliki kroga. Volumni se premikajo in imajo med sabo popolne elastične trke.
Osnovna zanka je sestavljena iz faze zaznave trkov med volumni, upoštevanja trkov in posledičnega
spreminjanja vektorjev objektov, premikanja objektov in optimizacije drevesa. Za učinkovitost
simulatorja je zelo pomembno, da je drevo čimbolj uravnoteženo, ter da je čim bolj plitko.
\section{Drevo objektov}

\textcolor{red}{Stvar bo treba še lektorirati, saj so povedi ponekod še dolge in okorne.}

Naša implementacija fizikalnega simulatorja za pospeševanje zaznave trkov uporablja BSP drevo.
Drevo se razdeljuje po premicah, ki so vzporedne koordinatnima osema.

\begin{wrapfigure}{r}{0.4\textwidth}
    
    \vspace{0.2cm}
    \centering

    \tikz[tree layout, grow'=down, level distance=11mm, sibling distance=3mm,
          nodes={draw,fill=cyan!40,circle,inner sep=2pt, scale=0.6}
    ]
    \node{NULL}
    child {node {NULL}
      child {node {NULL}
      }
      child {node[fill=red!50]{0x7F...}
        child{node[draw, rectangle, fill=white]{[... , ...]}}
      }
    }
    child {node {NULL}
      child {node {NULL}}
      child {node {NULL}}
    };
    \caption{Drevo z vozliščem z objekti}%
    \label{fig:drevo_z_buf}

\end{wrapfigure}

Drevo je sestavljeno iz vozlišč, ki so C strukture (izpis \ref{node_struct}). Za preprostejšo uporabo je drevo povezano
navzdol in navzgor -- vsako vozlišče ima kazalce na starševsko vozlišče in svoja dva potomca.
Razdeljevanje poteka s pomočjo dveh spremenljivk: smeri razdelitve (os $x$ ali os $y$) ter
poziciji razdelitve (primer: $1334.5$). Na tak način lahko funkcije, ki se spuščajo skozi drevo,
učinkovito določijo naslednje vozlišče, ki ga bodo obiskale\cite{klein_point_2004}. Objekti so v
drevesu shranjeni le v vozliščih, ki nimajo potomcev, torej so na dnu svoje veje drevesa.

\begin{lstlisting}[float, caption={Struktura vozlišča}, label=node_struct, language=C]
    struct treeNode {
        struct treeNode* left;
        struct treeNode* right;
        struct treeSplit split;
    //  vsebina strukture treeSplit:
    //      unsigned isx;
    //      double value;
        struct treeNode* up;
        object* buf; // NULL -> list
        uint64_t places; //bitmask: prazen = 1, zaseden = 0
        rect_llhh bindrect; //okvir, ki ga zaseda vozlišče
        unsigned level; // kako globoko v drevesu je vozlišče
        uint16_t flags;
        uint16_t optindex;
    };
\end{lstlisting}

Poleg povezav z ostalimi vozlišči in razdeljevanja ima vsako vozlišče tudi možnost kazalca (\lstinline|buf|) na
del spomina, kjer so shranjeni objekti. Ta kazalec ima tudi to praktično lastnost, da učinkovito
pokaže, ali je vozlišče na dnu svoje veje drevesa. Vključen je tudi pravokotnik \lstinline|bindrect|, ki opisuje površino,
ki jo pokriva vozlišče v simulaciji. Z njegovo pomočjo lahko ugotovimo, ali je treba objekt premestiti v drugo vozlišče.
Pomemben del vsakega vozlišča je bitna maska \lstinline|places|, ki označuje lokacijo vsakega objekta v kosu
pomnilnika, kjer so shranjeni. Na ta način lahko z eno bitno operacijo preverimo tudi, ali je v vozlišču
preveč objektov, in ga je treba razdeliti. Z procesorskim ukazom \lstinline|popcount| lahko tudi hitro
preštejemo objekte.
Poleg tega je v strukturi tudi bitna maska \lstinline|flags|, ki omogoča označevanje vozlišča, če ima
kakšen poseben status.

\subsection{Iskanje po drevesu}

Hitro najti objekt le s pomočjo njegovih koordinat je celoten razlog, zaradi katerega
sploh potrebujemo BSP drevo, zato je gladka in učinkovita implementacija iskalnih algoritmov
ključnega pomena. Obstajata dve vrsti iskanja:
Iskanje točke (objekta), ter iskanje območja (pravokotnika). Iskanje točke je pomembno za
vstavljanje in ponovno vstavljanje objektov v drevo, iskanje območja pa za zaznavo trkov.


Iskanje točke na izpisu \ref{find_parent_node} sicer ni rekurzivno, a pri
iskanju pravokotnika se srečamo z več potencialnimi zadetki.
\begin{samepage}
Iskanje v drevesu poteka na preprost način (prikazan tudi na izpisu \ref{find_parent_node}):
    \begin{enumerate}
        \item Začnemo pri korenu drevesa.
        \item Preverimo, ali ima trenutno vozlišče kazalec na objekte.
        \item Če kazalec ima, potem vrnemo kazalec na vozlišče.
        \item Če kazalca nima:
        \begin{itemize}
            \item Preverimo, po kateri koordinati se razdeljuje vozlišče.
            \item Razdelitev primerjamo z relevantno koordinato objekta.
            \item Glede na primerjavo se premaknemo na levega ali desnega potomca.
            \item Vrnemo se na korak 2.
        \end{itemize}    
    \end{enumerate}
\end{samepage}

Poleg iskanja točke je za korektno delovanje simulatorja pomembno tudi iskanje pravokotnika. Pri tem postopku
sledimo podobnemu algoritmu, a ob vsaki razdelitvi vozlišča preverimo, ali se pravokotnik tudi razdeli. V tem
primeru je treba rekurzivno poklicati funkcijo na obeh potomcih, saj oba vsebujeta del pravokotnika. Zaradi takih
situacij se pogosto zgodi, da iskanje pravokotnika vrne kazalce na več vozlišč, zato moramo upoštevati možnost
poljubnega števila zadetkov.
\subsection{Grajenje drevesa}

\begin{wrapfigure}{r}{0.6\textwidth}
    \centering
    \includestandalone{snips_pics/unbal_tree}
    \caption{Neuravnoteženo drevo, če vstavljamo objekte, urejene po koordinatah.}%
    \label{fig:drevo_unbal}
\end{wrapfigure}

Za grajenje dreves, ki organizirajo prostor, je veliko pristopov. Cilj vsakega pristopa je ponavadi čimbolj
uravnoteženo drevo. Večina jih je prilagojenih na to, da se objekte, ki jih organiziramo, bere v določenem
vrstnem redu. Ker je simulator, ki smo ga implementirali, namenjen le raziskovanju, lahko izberemo zelo 
preprost, a učinkovit način ustvarjanja uravnoteženega drevesa, vstavljanje naključnih objektov.


Ker se lahko zanesemo, da bo naključnost poskrbela za enakomerno, nesekvenčno vstavljanje objektov, se lahko
izognemo slabo zgrajenim, neuravnoteženim drevesom, kot je drevo na sliki~\ref{fig:drevo_unbal}.

\begin{lstlisting}[float, caption={Iskanje vozlišča, ki vsebuje objekt}, label=find_parent_node, language=C]
    treeNode* tree_findParentNode(objTree* tree, object* obj){
        treeNode* currentNode = tree->root;
        while(1) {
            if(currentNode->buf) return currentNode; // smo na dnu
            switch(currentNode->split.isx) {
                case 1: // x split
                    if(obj->x < currentNode->split.value){
                        currentNode = currentNode->left;
                    }
                    else {
                        currentNode = currentNode->right;
                    }
                    break;
                case 0: // y split
                    if(obj->y < currentNode->split.value){
                        currentNode = currentNode->left;
                    }
                    else {
                        currentNode = currentNode->right;
                    }
            }
        }
    }
\end{lstlisting}


Če hočemo objekt vstaviti v drevo, moramo najprej najti pravo vozlišče, nato preveriti,
ali je slučajno polno, in v tem primeru vozlišču dodati dva potomca in jima razdeliti objekte.
O tem bomo več povedali pod naslovom Deljenje vozlišč.

Objekti so v svojem kosu pomnilnika razporejeni nepredvidljivo. Razlog za to je, da se nekateri
morali prestaviti v drugo vozlišče zaradi svoje pozicije zunaj pravokotnika, ki ga pokriva
njihovo prejšnje vozlišče.
\subsubsection{Zapis objektov v pomnilniku}
Vsako vozlišče, ki lahko vsebuje objekte, ima kazalec na kos pomnilnika, v katerega lahko spravimo
točno določeno maksimalno število objektov. Da se izognemo nepotrebnim, časovno potratnim dostopom
do njih, je treba nekaj informacij o objektih zapisati že v strukturo vozlišča.

Dober zapis dosežemo tako, da si zapišemo, katera mesta v kosu pomnilnika z objekti
so polna in katera so prazna. V ta namen uporabljamo \lstinline|uint64_t places|, ki je del strukture 
\lstinline|treeNode|. S pomočjo takega 64-bitnega števila lahko z branjem in pisanjem posameznih bitov
spremljamo zasedenost kosa pomnilnika do največ 64 objektov, kar je več kot dovolj za naše potrebe. 
Nekaj osnovnih operacij je opisanih v izpisu~\ref{bitmask_examples}:
\begin{lstlisting}[caption={Uporaba bitne maske za objekte}, label=bitmask_examples, language=C]
    // preveri, ali je na indeksu N objekt
    !(node->places & (1UL << N))

    // preveri, ali je cel kos pomnilnika popisan.
    // OBJBUFSIZE je največje število objektov.
    !(node->places & ((1UL << OBJBUFSIZE) -- 1)) 

    // indeks prvega prostega mesta
    __builtin_ctzll__(node->places)

    // indeks prvega objekta
    __builtin_ctzll__(~node->places)

    // koliko je objektov: redko uporabljeno
    __builtin_popcountl__(~node->places)

    // zapišemo, da je prostor N zaseden
    node->places &= ~(1UL << N)

    // zapišemo, da je prostor N prost
    node->places |= (1UL << N)

\end{lstlisting}
\subsubsection{Alokacija pomnilnika}
Ker ima ta program dokajšnje zahteve po pomnilniškem prostoru, je smiselno namesto sistemskega
alokatorja \lstinline|malloc(), calloc(), ...| uporabiti lasten, preprostejši alokator.
Ker se mora sistemski alokator spoprijemati s težavami kot je fragmentacija pomnilnika, sproščanje
prej alociranih delov pomnilnika in mnogimi drugimi možnostmi, mora zelo pazljivo
ravnati z alokacijami in izvajati cel kup operacij, ki nam jih pravzaprav znotraj konteksta našega simulatorja
ni treba izvajati. Zato je smiselno, da implementiramo zelo preprost linearni alokator, ki nam lahko služi kot orodje
za spremljanje količine porabljenega pomnilnika, poleg tega pa tudi zelo pospeši alokacijo novih struktur.
Implementirali smo linearni alokator, ki v zakulisju alocira velike kose pomnilnnika s pomočjo
funkcije \lstinline|calloc()|, koščke teh alokacij pa potem daje funkcijam, ki rabijo pomnilnik.
Ker naš simulator vse alocirane kose pomnilnika ponovno uporabi, nam na primer sploh ni treba implementirati
\lstinline|free()|, ki ga sistemski alokator mora imeti. Ker smo simulator sprogramirali na tak način,
si bodo alokacije znotraj našega alokatorja vedno sledile ena za drugo, brez lukenj. Če bi si na primer želeli
podatke serializirati take kot so, bi lahko vzeli alokatorjev seznam vseh večjih kosov pomnilnika, jih zlepili
skupaj in zapisali v datoteko, kar je veliko preprosteje, kot če bi imeli nešteto majhnih sistemskih alokacij.
\subsection{Deljenje vozlišč}
Ko število objektov v določenem vozlišču doseže svojo maksimalno vrednost, moramo to vozlišče
razdeliti na dve. To zahteva alokacijo struktur za dve novi vozlišči in alokacijo vsaj enega novega
kosa pomnilnika za objekte. Poleg tega je treba vsaj en del objektov prestaviti v drugo vozlišče.
Proces deljenja vozlišča je torej relativno časovno potraten postopek, ki si ga ne želimo izvajati
preveč pogosto. Tako se poskušamo izogniti čim več operacijam s pomnilnikom, s čimer prihranimo
čas in količino porabljenega pomnilnika.

Preden se spustimo v detajle deljenja vozlišč, se moramo spomniti, da iskalne funkcije preverjajo,
ali ima vozlišče objekte, samo na podlagi njegovega kazalca \lstinline|buf|. To torej pomeni, da mora vozlišče,
ki ga delimo, na koncu imeti ta kazalec nastavljen na nič.

Cilj deljenja je, da pod vozliščem ustvarimo dva potomca, ki imata vsak svoj del objektov. Med deljenjem
je treba prilagoditi tudi bitne maske \lstinline|places| vozlišča in obeh potomcev.

Naiven način deljenja vozlišča bi bil lahko:
\begin{itemize}
    \item Alociramo dve novi vozlišči, ki bosta potomca.
    \item Alociramo dva nova kosa pomnilnika za njune objekte.
    \item Izračunamo mediano x/y koordinat objektov (odvisno od lokacije v drevesu).
    \item Zapišemo mediano v \lstinline|split| originalnega vozlišča.
    \item Na podlagi mediane razdelimo objekte med oba potomca. Med kopiranjem
    posodabljamo bitno masko \lstinline|places| obeh potomcev.
    \item Originalnemu vozlišču odstranimo kos pomnilnika z objekti.
\end{itemize}
Ta način zahteva dve alokaciji in eno odstranitev kosov pomnilnika. Poleg tega moramo tudi prekopirati
vse objekte iz originalnega kosa pomnilnika v pomnilnik od obeh potomcev, kar zahteva toliko operacij, kot
je maksimalno število objektov v vozlišču.

Bolj pameten način deljenja vozlišč pa bi potekal tako:
\begin{itemize}
    \item Alociramo dve novi vozlišči, ki bosta potomca.
    \item Alociramo en nov kos pomnilnika za objekte.
    \item Izračunamo mediano x/y koordinat objektov (odvisno od lokacije v drevesu).
    \item Na podlagi mediane prestavimo $ \frac{n}{2}$ objektov iz originalnega kosa pomnilnika
    v drugi kos pomnilnika. Med prestavljanjem posodabljamo \lstinline|places| vozlišča in njegovega potomca.
    \item Iz originalnega vozlišča prestavimo kazalec na njegov pomnilnik na njegovega potomca.
    \item Iz originalnega vozlišča prekopiramo bitno masko \lstinline|places| v potomca, ki je podedoval tudi pomnilnik.
\end{itemize}
Drugi način zahteva eno alokacijo in samo $\frac{n}{2}$ operacij premikanja. To pomeni, da je postopek
dvakrat hitrejši, kar je zelo dobrodošla sprememba.

Sama funkcija za deljenje vozlišč je zelo dolga, zato si bomo ogledali samo najpomembnejši
del, to je prestavljanje objektov. Potrebujemo še nekaj konteksta glede kode: \lstinline|parent|
je vozlišče, ki ga cepimo. \lstinline|dest| je vozlišče, kamor kopiramo objekte. Naloga vrstice
8 je, da standardizira operacijo za obe mogoči razdelitvi: $x$ in $y$. Strojna koda, v katero se to
prevede, je enaka, kot če bi napisali \lstinline|parent->buf[i].x|, a s pomembno razliko, da izbira
polja \lstinline|x| ali \lstinline|y| v strukturi ni vnaprej določena med prevajanjem.

\begin{lstlisting}[caption={Razdeljevanje objektov med potomcema}, label=balance_buffers, language=C]
if(parent->split.isx) 
    coordoffset = offsetof(object, x);
else 
    coordoffset = offsetof(object, y);
unsigned destwritten = 0;
for(unsigned i = 0; i < OBJBUFSIZE; i++){
    //first value is the x or y field of the i-th struct in buf
    if(*(double*)( (char*)&(parent->buf[i]) + coordoffset )
     > parent->split.value)
        {
        memcpy(dest->buf+destwritten, parent->buf+i, sizeof(object));
        memset(parent->buf + i, '\0', sizeof(object));
        parent->places |= 1UL << i; // prazen
        dest->places &= ~(1UL << destwritten); // zaseden
        destwritten++;
    }
}
\end{lstlisting}
\section{Delovanje simulatorja}
\subsection{Zaznava trkov}
Vse funkcije in mehanizme, ki smo si jih ogledali v prejšnjih poglavjih, lahko sedaj uporabimo
za osrednji del našega simulatorja, zaznavo trkov.
Preden začnemo z zaznavanjem trkov, moramo spoznati oobjekte, s katerimi se ukvarjamo.
\begin{lstlisting}[caption={Struktura objekta}, label=object_struct, language=C]
struct object {
    double x;
    double y;
    speed v;
    float s;
    unsigned m;
    struct object* hit; // kazalec na objekt v kontaktu
    unsigned long id;
    unsigned flags;
};
\end{lstlisting}
Objekt ima poleg koordinat \lstinline|x| in \lstinline|y| tudi hitrost \lstinline|v| (vektor dveh \lstinline|double|),
velikost \lstinline|s| (polmer kroga, ki predstavlja objekt) ter maso \lstinline|m|. Od nefizikalnih podatkov ima
identifikacijsko številko \lstinline|id| in bitno masko \lstinline|flags| za dodatne označbe.

V strukturi je tudi \lstinline|hit|, kazalec na drug objekt. Funkcija za iskanje trkov v ta kazalec zapiše naslov
drugega objekta, ki je trčil z njim. Funkciji, ki procesirata podatke o trkih, prebereta
ta kazalec in izračunata nove hitrosti in pozicije objektov. To je zelo pomanjkljiv sistem, saj ne omogoča,
da en objekt trči z več drugimi hkrati. To včasih vodi do fizikalno nepravilnega dogajanja, ki je pa redko
in ne vpliva zelo na našo simulacijo. Bolj smiseln pristop bi bil, da bi implementirali čakalno vrsto, kamor
bi funkcija za iskanje trkov vstavljala kazalce na objekte, funkciji ki procesirata podatke o trkih pa bi jih
upoštevali pri simulaciji.
\begin{lstlisting}[float, breaklines=true, postbreak=\mbox{\textcolor{purple}{$\hookrightarrow$}\space}, caption={Zaznava trkov}, label=hit_detect, language=C]
static int hit_flagObjects_aux(objTree* tree, treeNode* node){
  if(!node->buf){//rekurzija, če ni objektov
    hit_flagObjects_aux(tree, node->left);
    hit_flagObjects_aux(tree, node->right);
    return 0;
  }
  for(unsigned i = 0; i < OBJBUFSIZE; i++){
    if(node->buf[i].s && !node->buf[i].hit){      
      // iskanje pravokotnika okoli objekta
      unsigned numResults = tree_findRect_shallow(
        tree,
        node,
        (rect_ofex){ //pravokotnik okoli objekta
          ...
          },
        tree->searchbuf
      );
      for(unsigned bufindex = 0; bufindex < numResults; bufindex++){
        object* buf = tree->searchbuf[bufindex];
        for(unsigned j = 0; j < OBJBUFSIZE; j++){
          //ne smemo zadeti sami sebe
          if(buf + j == node->buf + i) continue; 
          if(buf[j].s && !buf[j].hit){
            double dist = SQUARE(buf[j].x - node->buf[i].x) + SQUARE(buf[j].y - node->buf[i].y);            
            if(dist < SQUARE(node->buf[i].s + buf[j].s)){
              node->buf[i].hit = buf + j;
              buf[j].hit = node->buf + i;
              break;
  }}}}}}
  return 0;
}
\end{lstlisting}

Zaznavanje trkov poteka kot rekurzivna funkcija, ki gre skozi vse objekte v drevesu. Za vsakega poišče vozlišča,
ki vsebujejo njegovo območje vpliva -- kvadrat okoli objekta, nato pa za vse objekte v njih preveri, ali so z njim
trčili. Celotna funkcija je v izpisu~\ref{hit_detect}.

Ker je iskanje trkov samo po sebi kvadratna operacija, bomo gotovo potrebovali dve zanki \lstinline[language=C]|for|.
Prva zanka bo šla skozi objekte v vsakem vozlišču, druga zanka pa bo šla skozi rezultate iskanja območja vpliva okoli
objekta (klic funkcije \lstinline|tree_findRectShallow|). Ker so rezultati vozlišča in ne objekti, moramo iti skozi vse
objekte v vsakem vozlišču, ki ga je našla iskalna funkcija. Ponavadi je to samo eno (lastno vozlišče) ali kvečjemu dve
vozlišči (objekt je ob robu lastnega vozlišča).

Večje število rezultatov dobimo samo takrat, ko je objekt v bližini trojne ali četverne meje,
torej v kotu lastnega vozlišča. Skozi veliko vozlišč moramo tudi takrat, ko je koncentracija 
objektov na določenem kraju velika. Simulator je v tem primeru primoran ustvariti veliko novih vozlišč, ki so 
majhna in vedno bolj primerljiva z območji vpliva okoli objektov. To povzroči, da območja vpliva okoli objektov
pokrivajo vedno več vozlišč.

Funkcija je sestavljena iz treh delov. Prvi del je rekurzija. Če trenutno vozlišče nima objektov, funkcija kliče samo sebe
z obema potomcema vozlišča, ki ga trenutno obravnava. Ko funkcija pride do vozlišča z objekti, se začne prva zanka \lstinline[language=C]{for}.
V njej funkcija preveri, ali objekt obstaja, in ali je v istem koraku že trčil z drugim objektom. Nato poizve, katera vozlišča
vsebujejo določeno območje vpliva okoli objekta. Z drugo zanko \lstinline[language=C]{for} za vsak rezultat izvede še eno zanko \lstinline[language=C]{for},
ki preveri, ali kakšen objekt v tem vozlišču trči z našim objektom. Na koncu preveri, ali objekta trčita. V tem primeru
jima kazalce \lstinline[language=C]{hit} nastavi tako, da kažeta eden na drugega. To zagotovi, da bo funkcija vedela,
ali je ta dva objekta že obravnavala. Te kazalce funkcija \lstinline[language=C]{vector_update()} nato uporabi za to, da
popravi njune vektorje. 
\subsection{Posodabljanje objektov}
Vse zaznane trke moramo v vsakem koraku simulacije tudi upoštevati. Najprej moramo popraviti njihove hitrosti, da bodo
v skladu z morebitnimi trki, potem pa moramo tudi preveriti, ali se je kakšen objekt slučajno premaknil čez rob svojega
vozlišča. V tem primeru moramo objekt ponovno vstaviti v drevo in rešiti vse morebitne težave, ki pri tem nastanejo.
Poleg tega lahko za hitrejše delovanje simulatorja tudi preverimo, ali je kakšno vozlišče skoraj prazno, in ga postavimo
v čakalno vrsto za reciklažo. Temu se bomo bolj posvetili kasneje.

Funkcija za posodabljanje objektov se imenuje \lstinline[language=C]{vector_update()}.

Začne se z rekurzivnim delom, kjer preveri, ali je na vozlišču z objekti. V primeru da ni, pokliče samo sebe
s potomci svojega vozlišča.
\newpage
\begin{lstlisting}[caption={vector\_update() -- Rekurzivni del}, label=vectorupdate1, language=C]
int vector_update_aux(objTree* tree, treeNode* node){
  if(!node->buf){
    nullbuf:
    vector_update_aux(tree, node->left);
    vector_update_aux(tree, node->right);
    return 0;
  }
\end{lstlisting}
Preden se začnemo ukvarjati z objekti, lahko preverimo, ali je trenutno vozlišče primerno za reciklažo.
V tem primeru je kriterij, da ima $ \frac{1}{4}$ ali manj objektov, kot bi jih lahko imelo.
Če se to zgodi, potem vozlišče funkcija \lstinline{tree_OptqueueSubmit()} vstavi v čakalno vrsto.
Pred tem je seveda treba preveriti, ali je to vozlišče že v čakalni vrsti, saj nočemo imeti velikega števila
kopij.
\begin{lstlisting}[caption={vector\_update() -- Reciklaža}, label=vectorupdate2, language=C]
#define NFLAG_IN_OPTQUEUE 1U

...

#if RECYCLE == 1
unsigned numObjects =__builtin_popcountl(~(node->places)); //število ničel (polnih mest)
if(numObjects <= ((unsigned)OBJBUFSIZE / 4U)){
  if(!(node->flags & NFLAG_IN_OPTQUEUE)){
      tree_OptqueueSubmit(tree, node);
  }
}
#endif
\end{lstlisting}
Sledi popravljanje hitrosti objektov, da ustrezajo situaciji po trku. Sledimo formuli za elastičen odboj
v dveh dimenzijah. Treba je izračunati hitrost centra mase obeh objektov, nato pa hitrosti obeh objektov
glede na center mase obrniti za 180°.

Hitrost centra mase je: 
\[
v_{cm} = \frac{m_A v_A + m_B v_B}{m_A + m_B}
\]
Zdaj moramo izračunati hitrost objekta A glede na hitrost centra mase:
\[
v = v_A - v_{cm}
\]
Potem moramo zamenjati smer hitrosti:
\[
v = v\cdot(-1)
\]
In potem nazaj prišteti hitrost centra mase, da dobimo končno hitrost objekta:
\[
v_{A'} = v + v_{cm}
\]
Ta postopek se da poenostaviti. Formula za hitrost centra mase ostane enaka, lahko pa združimo vse ostale korake:
\[
v_{A'} = - v_A + 2\cdot v_{cm}
\]
V tem izpisu kode je ta formula implementirana. Za lažjo berljivost so matematični izrazi izločeni.
\begin{lstlisting}[caption={vector\_update() -- Upoštevanje trkov}, label=vectorupdate3, language=C]
for(unsigned i = 0; i < OBJBUFSIZE; i++){
  object* obj = node->buf + i;
  if(obj->s){ //če objekt nima velikosti, potem ne obstaja 
              //(vsi veljavni objekti imajo velikost)
    if(obj->hit) {
      speed v_cm = { .x ..., .y = ...};
      obj->v = (speed){.x = ..., .y = ...};
      obj->hit->v = (speed){.x = ..., .y = ...};
      //ponastavi kazalce
      obj->hit->hit = 0;
      obj->hit = 0;
    }
\end{lstlisting}
Ostane nam še nekaj preverjanja, ali je objekt slučajno prekoračil kakšno mejo -- bodisi fizično prepreko okoli simulacije,
bodisi mejo vozlišča. Če je zunaj prepreke okoli simulacije, se mu obrne vektor tako, da se bo odbil. Če je prešel
mejo svojega vozlišča, potem ga moramo ponovno vstaviti v drevo.
\begin{lstlisting}[caption={vector\_update() -- Odboji}, label=vectorupdate4, language=C]
if((obj->x > g_rightborder && obj->v.x > 0) || (... && ...))
  //BOUNCE omogoči nastavljanje elastičnosti odboja od stene
  obj->v.x *= -BOUNCE; 
if((...) || (...))
  obj->v.y *= -BOUNCE;
if((obj->x > node->bindrect.highhigh.x || obj->x < node->...)){
  tree_insertObject(tree, obj);
}
\end{lstlisting}
Tej funkciji sledi še funkcija \lstinline[language=C]{scalar_update()}, ki objekte premakne v skladu z
njihovimi vektorji hitrosti. Obe funkciji bi se dalo združiti, a zaradi praktičnih razlogov tega še nismo
izvedli.

\subsection{Reciklaža vozlišč}
Ko se simulacija začne, se objekte naključno vstavlja v prostor. Drevo je na začetku simulacije
skoraj popolnoma uravnoteženo in najučinkovitejše. Ko se objekti premikajo skozi drevo, se pogosto zgodi,
da v prej skoraj popolnoma zasedeno vozlišče vstopi dovolj objektov, da se mora vozlišče razdeliti. Čez nekaj
časa pa ti objekti nadaljujejo svojo pot in zapustijo oba potomca tega vozlišča. Drevo je zdaj globlje, 
tega prostora pa objekti sploh ne potrebujejo. To upočasni delovanje simulatorja z odvečnimi, nepotrebnimi
vozlišči.
\begin{wrapfigure}{r}{0.5\textwidth}
    \centering
    \tikz[tree layout, grow'=down, level distance=11mm, sibling distance=3mm,
          nodes={draw,fill=cyan!40,circle,inner sep=2pt, scale=0.6}
    ]
    \node{NULL}
    child {node[fill=red!50, inner sep=9pt]{\large2}
      child{node[draw, rectangle, fill=white]{[o4, o5]}}
    }
    child {node[fill=red!50, inner sep=9pt]{\large3}
      child{node[draw, rectangle, fill=white]{[o1, o2, o3]}}
    };
    \caption{Drugi pristop -- začetno stanje}%
    \label{fig:drevo_reciklaza1}
\end{wrapfigure}

Rešitev za ta problem je, da prazna ali skoraj prazna vozlišča sproti združujemo s sosednjimi, dokler
nimamo vozlišča z dovolj visokim številom objektov. Na koncu moramo iz dveh potomcev, ki imata vsak
nekaj objektov, dobiti eno vozlišče z objekti iz obeh potomcev. Kaj se pa zgodi s potomcema? Izbiramo
lahko med dvema pristopoma. Prvi pristop je, da alociramo nov kos pomnilnika za starša, vanj skopiramo
vse objekte, oba potomca z njunima kosoma pomnilnika pa izbrišemo. Ta pristop vzame delo iz rok algoritma
za združevanje, in ga preloži na alokator. Ta mora zdaj, če se hočemo izogniti prekomerni uporabi pomnilnika,
slediti vsem prostim kosom znotraj svojega bloka pomnilnika, in jih ob novih alokacijah vrniti v obtok.
Prednost tega pristopa je, da dolgoročno ob podobnem številu objektov količina porabljenega pomnilnika ostane
enaka. Na žalost to pomeni, da bo alokator ob vsaki alokaciji porabil več časa. Ta pristop je uporaben,
če imamo zelo velik prostor ter majhno število objektov, za katere je majhna verjetnost, da se bodo spet
vrnili tja, kjer so bili že prej.
\begin{wrapfigure}{r}{0.5\textwidth}
    \centering
    \tikz[tree layout, grow'=down, level distance=11mm, sibling distance=3mm,
          nodes={draw,fill=cyan!40,circle,inner sep=2pt, scale=0.6}
    ]
    \node[fill=red!50, inner sep=9pt]{\large5}
    child {node[fill=black!10, inner sep=9pt]{\large0}
      child{node[draw, rectangle, fill=white]{[/]}}
    }
    child{node[draw, rectangle, fill=white]{[o1, ... o5]}
    }
    child {node[fill=black!10,]{NULL}
    };
    
    \caption{Drugi pristop -- končno stanje}%
    \label{fig:drevo_reciklaza2}
\end{wrapfigure}

Drugi pristop je, da sploh ne izbrišemo potomcev ki jih ne potrebujemo več. Namesto tega jih ohranimo povezane s
starševskim vozliščem. Ker starševsko vozlišče potrebuje kos pomnilnika za objekte, lahko enemu potomcu odvzamemo
ta pomnilnik in ga damo starševskem vozlišču. Ker si želimo čim krajši postopek kopiranja objektov, kos vzamemo tistemu vozlišču,
ki ima več objektov. To pomeni, da bomo zmanjšali količino kopiranja na najmanjšo mogočo vrednost. Ker iskalne funkcije
zaznajo vozlišče z objekti na podlagi kazalca na objekte, lahko starševsko vozlišče ostane povezano s potomcema.
Ko bo naslednjič v njem preveč objektov, lahko funkcija za razdeljevanje uporabi prazen kos pomnilnika od enega
od potomcev, ter oba potomca. Tako ji ni treba alocirati novih vozlišč ali kosov pomnilnika za objekte.
Ta način je hitrejši, a povzroči prekomerno uporabo pomnilnika, če se v prej polna vozlišča objekti ne vračajo.

Za naš simulator smo izbrali drugi pristop, ker imamo majhno, omejeno območje simulacije. Prvi pristop bi bil
boljši za simulacije z veliko praznega prostora in majhnimi, koncentriranimi skupinami objektov.

Drugi pristop k reciklaži smo implementirali v funkciji \lstinline[language=C]{tree_optimizeNodes()}.
Funkcija uporablja čakalno vrsto, v kateri so vozlišča, primerna za reciklažo. To čakalno vrsto spotoma polni
kar funkcija \lstinline[language=C]{vector_update()} (izpis~\ref{vectorupdate2}), saj je smiselno čim bolj zmanjšati število rekurzivnih
prehodov skozi drevo vsak korak. To namreč zahteva, da se celotno drevo naloži v predpomnilnik in procesira,
zato je dobro, da vse, kar mislimo narediti z vozlišči in objekti naredimo v čim manj prehodih skozi drevo.
Tak pristop sicer ni v skladu z ideali modularnega programiranja, a ta problem več kot odtehtajo časovne pridobitve.

Oglejmo si del funkcije, kjer se objekti kopirajo. Implementiran je na tak način, da se zanka izvede samo tolikokrat,
kolikor je objektov, ki jih moramo kopirati. To je časovno učinkovit način kopiranja, saj do podatkov dostopamo zelo
malokrat. Za razumevanje tega izseka kode si lahko pomagamo z izpisom~\ref{bitmask_examples}.
\begin{lstlisting}[caption={tree\_optimizeNodes() -- Kopiranje objektov}, label=optimizenodes, language=C]
for(unsigned j = 0; j < numObjects_src; j++){
  //prvi objekt
  srcoffset =  __builtin_ctzl(~(src->places));
  //prvi prazen prostor
  destoffset = __builtin_ctzl(dest->places);
  memcpy(dest->buf + destoffset,
         src->buf + srcoffset,
         sizeof(object)
  );
  dest->places &= ~(1UL << destoffset);
  src->places |= (1UL << srcoffset);
}
\end{lstlisting}
%****************************************************************************************
%                                 EKSPERIMENTALNI DEL                                   *
%****************************************************************************************
\newpage
\section{Eksperimentalni del}
\subsection{Število objektov}

\begin{wrapfigure}{r}{0.5\textwidth}
    \adjustbox{max width=0.5\textwidth}{
        \includestandalone{numobjects}
    }
    \caption{Vpliv števila objektov na hitrost simulatorja}
\end{wrapfigure}

Uporaba posebnih podatkovnih struktur je zelo pomembna za učinkovitost takih simulatorjev, kot je naš.
Ker je časovna učinkovitost eden od najpomembnejših kriterijev za razvoj, jo moramo tudi mi zelo temeljito
obravnavati. Preučili bomo nekaj faktorjev, ki pomembno vplivajo na hitrost izvajanja, in kako lahko prilagodimo
simulator, da bo čim bolj učinkovit. 

Glavni faktor, ki vpliva na hitrost izvajanja simulacije, je njena obsežnost. Cilj našega simulatorja je,
da zelo dobro prenese velike sisteme, ki jih mora simulirati. Uspešnost programa smo merili tako, da smo vedno
zagotovili enako gostoto objektov, saj bi sicer večja količina objektov pomenila več trkov na posamezen objekt,
kar bi skvarilo podatke. 

Merili smo čas, ki ga simulator porabi za simulacijo 1000 korakov pri danem številu objektov.
Objekti so bili po prostoru razporejeni naključno, da je bila povsod približno enaka gostota objektov. Imeli so
naključne hitrosti v naključne smeri. Njihove mase so bile enake, prav tako velikosti. 

Vidimo lahko, da graf ni linearen. Po okoli 50.000 objektih se bolj strmo vzpenja, poleg tega pa podatki
postanejo dosti manj predvidljivi. Ugotovili smo, da se to zgodi zaradi velikosti predpomnilnika procesorja, na
katerem je bil program testiran. Ko število objektov doseže določeno vrednost, je predpomnilnik zapolnjen in
morajo podatki prihajati iz glavnega pomnilnika. Po 80.000 objektih lahko vidimo, da čas narašča v skoraj
ravni črti. Ker uporabljamo BSP drevo, je teoretična časovna kompleksnost $O(n \log n)$, kjer je $n$ število objektov.
Vemo, da sta število objektov v vozlišču in gostota objektov konstantna. Iz teh dveh podatkov
lahko zaključimo, da se povprečen porabljen čas na objekt znotraj zaznave trkov ne spreminja, torej ima kompleksnost $O(n)$.
Od kod potem pride $\log n$?
Vsakič, ko iščemo sosede določenega objekta, moramo narediti poizvedbo skozi drevo. Ta poizvedba ima kompleksnost $O(\log n)$. 
Na grafu lahko ta faktor vidimo kot rahlo ukrivljenost navzgor, če seveda ignoriramo prehod s procesorjevega predpomnilnika na pravi pomnilnik.

\subsection{Velikost vozlišč}

\begin{wrapfigure}{r}{0.5\textwidth}
    \adjustbox{max width=0.5\textwidth}{
        \includestandalone{boxsizes}
    }
    \caption{Vpliv števila objektov v vozlišču na hitrost simulatorja}
\end{wrapfigure}

Število objektov v vozlišču je zelo pomembno za časovno učinkovitost programa. Končni iztek simulacije mora
biti neodvisen od notranjega delovanja simulatorja, zato moramo poskrbeti, da spreminjanje tega števila ne spremeni
dogajanja v simulaciji. Ta nastavitev vpliva na pogostost (in s tem časovno učinkovitost) dveh zelo pomembnih operacij:
Iskanja po drevesu ter iskanja znotraj vozlišč.

Iskanje po drevesu ima časovno kompleksnost $O(\log(n))$, kjer je $n$ število objektov v drevesu. Iskanje znotraj vozlišč
ima časovno kompleksnost $O(m^2)$, kjer je $m$ število objektov v vozlišču. Tukaj ne smemo pozabiti na rezultate iskanja po 
drevesu, saj imajo vozlišča z manj objekti večinoma manjšo površino kot tista z več objekti. Zato v primeru manjših vozlišč
na vsako poizvedbo dobimo povprečno več rezultatov. Seveda pa nižje število objektov v vozlišču pomeni manj objektov, za katere
moramo preveriti, ali so trčili, saj bolj natančno oklepajo območje vpliva okoli posameznega objekta. Na primer, če imamo
vozlišča z 20 objekti na vozlišče, potem bomo za objekt na meji med dvema vozliščema morali preveriti največ 39 kandidatov za trk.
Če je pa objektov na vozlišče 6, bo kandidatov največ 12.

Ne smemo pozabiti tudi na globino drevesa. Če je objektov v
vozliščih manj, bodo drevesa globlja. To pomeni več porabljenega časa na vsako iskanje območja v drevesu. Najmanj časa bo program
porabil takrat, ko sta oba vpliva skupaj najmanjša. Na vrednost najučinkovitejšega števila vplivata predvsem dva faktorja: zahtevnost
iskalne funkcije in zahtevnost preverjanja trkov. Če bi na primer imeli zahtevnejše preverjanje trka, bi bilo optimalno število
bližje 1. Ta premik bi se zgodil zato, ker je pri manjših vozliščih manj potencialnih kandidatov za trk, kar
zmanjša količino preverjanja za trke. Prav tako bi se ob manj učinkovitih funkcijah za iskanje optimalno število povečalo, saj
bi bilo v drevesu manj vozlišč. Zato bi bilo potem drevo plitkejše in bi bilo vozlišča v njem lažje najti. 

\newpage
\subsection{Koncentracija objektov}

\begin{wrapfigure}{r}{0.5\textwidth}
    \adjustbox{max width=0.5\textwidth}{
        \includestandalone{density}
        
    }
    \caption{Vpliv gostote objektov na hitrost simulatorja}
\end{wrapfigure}

Koncentracija objektov je popolnoma odvisna od začetnih pogojev simulacije. Nanjo kot razvijalci simulatorja
ne moremo vplivati, a je koristno premisliti, kako vpliva na učinkovitost in zakaj.

Koncentracija v tem primeru pomeni razmerje med površino objektov v simulaciji in med površino celotne simulacije.
Ima relativno majhen vpliv na učinkovitost, ki je pa še vedno relevanten. Vpliv je pomembnejši takrat, ko postanejo
površine objektov in vozlišč primerljive. Takrat začne iskanje območij dajati vedno več rezultatov, objekti se pogosteje
zaletavajo in pogosteje prehajajo med vozlišči. \textcolor{red}{<DODATEN GRAF: povprečna površina vozlišča v primerjavi z gostoto?>}

Trenutno je implementacija zastavljena tako, da je razreševanje trkov zelo preprosto, poleg tega se pa za trk preveri vsak
objekt. To pomeni, da bodo v vsakem primeru obravnavani vsi objekti, ne glede na število objektov, ki so trčili.
Zaradi take implementacije, ki je neučinkovita a preprosta, pravzaprav s temi podatki zaznavamo le to, koliko časa porabi
koda za popravljanje vektorjev objektov, ki so se odbili. To je zelo preprost postopek, zato spremembe v gostoti niso zelo pomembne
za učinkovitost programa.

Bolje napisan simulator bi zato imel veliko večje razlike pri spreminjanju gostote, saj bi bilo veliko
operacij izvedenih samo v primeru, da je objekt trčil. V tem primeru ne bi bilo treba na vsak korak prebrati vseh objektov,
ampak samo zelo majhno število objektov, ki so dejansko trčili med sabo.
\newpage
\subsection{Reciklaža vozlišč}
\begin{wrapfigure}{r}{0.45\textwidth}
    \adjustbox{max width=0.45\textwidth}{
        \includestandalone{recycling_yes_nograv}
    }
    \caption{Vklopljena reciklaža in izklopljena gravitacija}
    \vspace{0.5cm}
    \adjustbox{max width=0.45\textwidth}{
        \includestandalone{recycling_no_nograv}
    }
    \caption{Izklopljena reciklaža in izklopljena gravitacija}
\end{wrapfigure}
Reciklaža vozlišč je tehnika, ki lahko naredi veliko razliko, ali pa je sploh ne naredi. Njena učinkovitost
je zelo odvisna od dogajanja znotraj simulacije. Da bi predstavili to razliko v učinkovitosti, si bomo ogledali
učinkovitost simulatorja pri dveh različnih simulacijah, potem pa raziskali, v katerih primerih je reciklaža vozlišč efektivna.
\subsubsection{Breztežnost}
Prva situacija je sledeča: Objekte vstavimo na naključne pozicije v prostor in jim priredimo naključne hitrosti. Ko 
se zadenejo v steno simulacije, se popolnoma elastično odbijejo.

To pomeni, da bodo objekti skozi čas ostali približno enakomerno razporejeni po prostoru,
in ne bo nobenih trajnih praznih prostorov.


Če si ogledamo grafa, lahko vidimo, da reciklaža nima skoraj nobenega vpliva na učinkovitost simulatorja. To se zgodi zato, ker
v obeh primerih ni praznih območij, ki bi jih zasedalo preveč vozlišč. Ker so objekti približno enako gosto razporejeni povsod,
ni nobenega prostora, kjer bi reciklaža vozlišč prišla v poštev.


V obeh primerih lahko vidimo, da se učinkovitost simulatorja
počasi zmanjšuje, saj se prej območja z manj gosto razporeditvijo objektov prej ali slej dovolj napolnijo, da se kakšno vozlišče
razdeli.

\clearpage

\subsubsection{Gravitacija}

\begin{wrapfigure}{r}{0.45\textwidth}
    \adjustbox{max width=0.45\textwidth}{
        \includestandalone{recycling_no_grav}
    }
    \caption{Izklopljena reciklaža in vklopljena gravitacija}%
    \label{fig:recycling_no_grav}
    \vspace{0.5cm}
    \adjustbox{max width=0.45\textwidth}{
        \includestandalone{recycling_yes_grav}
    }
    \caption{Izklopljena reciklaža in vklopljena gravitacija}%
    \label{fig:recycling_yes_grav}
\end{wrapfigure}

Druga situacija ima v primerjavi s prvo dve zelo pomembni razliki: Na objekte deluje gravitacija, zato se odbijajo predvsem od spodnjega
roba simulacije, poleg tega pa odboj od robov ni popolnoma elastičen. To povzroči, da se vedno več objektov začne združevati na dnu prostora, nad
njimi pa je prostor vse bolj prazen.

Zaradi gravitacije ostane veliko vozlišč na vrhu praznih, na dnu pa se mora alocirati vedno več novih vozlišč. To pomeni, da se s časom
število vozlišč v drevesu močno poveča. To povečanje zelo slabo vpliva na učinkovitost simulatorja, 
kar je razvidno na sliki~\ref{fig:recycling_no_grav}.

Ker je v drevesu zdaj veliko trajno praznih vozlišč, lahko algoritem za reciklažo nemoteno deluje in drastično zmanjša število
neuporabljenih vozlišč. Na sliki~\ref{fig:recycling_yes_grav} lahko vidimo, da kljub ustvarjanju veliko novih vozlišč 
učinkovitost simulatorja ostane skoraj konstantna, saj se večina vozlišč na vrhu prostora reciklira in nima več vpliva na simulacijo.

Slabšanje učinkovitosti simulatorja skozi čas lahko pojasnimo tudi s tem, da zaradi večje gostote objektov na dnu objekti pogosteje
prehajajo med vozlišči. Poleg tega tudi pogosteje trčijo, kar tudi porabi nekaj časa. Zaradi tega lahko na sliki \ref{fig:recycling_yes_grav}
vidimo rahel naklon navzgor.
\newpage
\lstlistoflistings{}
\bibliographystyle{plain}
\bibliography{raziskovalna.bib}
% End Document
\end{document}
% =============================================================================================
% =============================================================================================
% =============================================================================================
% =============================================================================================
% =============================================================================================
% =============================================================================================
% Sections and Content
\begin{comment}
    \section{Game plan}
    Pojavi in eksperimenti, ki jih bom opisoval v raziskovalni nalogi:
    \begin{itemize}
        \item Vpliv velikosti škatel na hitrost simulacije
        \item Vpliv gostote predmetov na hitrost
        \item Vpliv reciklaže škatel na hitrost
        \item Vpliv linearnega alokatorja vs malloc()
        \item 
    \end{itemize}
    Kako bodo videti poglavja?
    \begin{itemize}
        \item Teoretični uvod (kompleksnost, drevesa -- k-d, BSP, ...)
        \begin{itemize}[label=$\diamond$, left=0.5cm]
            \item kompleksnost -- zakaj ne bruteforce?
            \item vrste dreves\cite{klein_point_2004}
            \item zakaj k-d?
            \item opis podatkovnih struktur (slikice?)
        \end{itemize}
        \item Opis projekta
        \item 
    \end{itemize}
    
    Seveda moramo pomisliti na prvi.
    
    In drugi odstavek v tem besedilu.
    \end{comment}


\section{Uvod}
Uvod v zaznavo trkov (najdi dober članek, ki govori o tem?)
\section{Pregled področja}
    \begin{itemize}
        \item kompleksnost in bruteforce
        \item zato je podatkovna struktura pomembna
        \item obstajajo knjižnice -- Kaj implementirajo?
        \item glavne podatkovne strukture, zakaj prav k-d drevo?
        \item razložim graf kompleksnosti ($n^2$ vs $n*\log(n)$)
    \end{itemize}
\section{Arhitektura implementacije}
    \subsection{Zaznava trkov}
    Kako delujejo moje funkcije?
    \subsection{Pomožne funkcije}
    Premikanje objektov naokrog in druge dejavnosti
    \subsection{Testno okolje}
    Kako merim podatke?
\section{Eksperimentalni del}
    Optimizacije (seznam in obrazložitev, eksp. podatki) TikZ za grafe
    \begin{itemize}
        \item Vpliv velikosti škatel na hitrost simulacije
        \item Vpliv gostote predmetov na hitrost
        \item Vpliv reciklaže praznih škatel na hitrost (WIP)
        \item Vpliv linearnega alokatorja vs malloc()
        \item Vpliv gravitacije (prostorskega zgoščevanja škatel) na hitrost
    \end{itemize}
\section{Zaključek}
Kaj sem naredil in pot naprej, primerjava z obstoječimi implementacijami
