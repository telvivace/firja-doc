\documentclass[a4paper,12pt]{article}

% Packages for headers, footers, and general formatting
\usepackage[utf8]{inputenc} % Encoding
\usepackage{fancyhdr}       % Headers and footers
\usepackage{enumitem}       % For advanced lists
\usepackage[luatex]{hyperref}       % For clickable URLs, citations etc
\usepackage[usenames,dvipsnames]{color}%has to be called before tikz
\usepackage{tikz}
\usetikzlibrary{graphs,graphdrawing}
\usegdlibrary{trees}
\usepackage{color}
\usepackage[english,slovene]{babel} %language support
\usepackage{listings}
\usepackage[mode=buildnew]{standalone}
% Header and Footer Setup
\pagestyle{fancy}
\fancyhf{} % Clear all header and footer fields
%\fancyhead[R]{Your Header on the Left} 

\fancyhead[C]{Optimizacija zaznave trkov v 2D}
\fancyfoot[C]{\thepage}
\setlength{\textwidth}{16cm}
\setlength{\oddsidemargin}{0cm}
\setlength{\headwidth}{\textwidth}
\linespread{0.5}

% Begin Document
\begin{document}
\makeatletter
\addto\captionsslovene{
  % handle hyperref's autoref 
  \renewcommand\equationautorefname{enačba}%
  \renewcommand\footnoteautorefname{opomba}%
  \renewcommand\itemautorefname{alineja}%
  \renewcommand\figureautorefname{slika}%
  \renewcommand\tableautorefname{tabela}%
  \renewcommand\partautorefname{del}%
  \renewcommand\appendixautorefname{priloga}%
  \renewcommand\chapterautorefname{poglavje}%
  \renewcommand\sectionautorefname{razdelek}%
  \renewcommand\subsectionautorefname{podrazdelek}%
  \renewcommand\subsubsectionautorefname{podpodrazdelek-section}%
  \renewcommand\paragraphautorefname{odstavek}%
  \renewcommand\subparagraphautorefname{del odstavka}%
  \renewcommand\FancyVerbLineautorefname{vrstica}%
  \renewcommand\theoremautorefname{teorem}%
  \renewcommand\pageautorefname{stran}%
  % handle algorithm2e
  % \renewcommand{\listalgorithmcfname}{Seznam algoritmov}%
  % \renewcommand{\algorithmcfname}{Algoritem}%
  % \renewcommand{\algorithmautorefname}{\algorithmcfname}%
  % \renewcommand{\algorithmcflinename}{Vrstica}%
  % \renewcommand{\algocf@typo}{}%
  % \renewcommand{\@algocf@procname}{Procedura}%
  % \renewcommand{\@algocf@funcname}{Funkcija}%
  % \renewcommand{\procedureautorefname}{\@algocf@procname}%
  % \renewcommand{\functionautorefname}{\@algocf@funcname}%
  % \renewcommand{\algocf@languagechoosen}{slovene}%
  % handle lstlistings
  \renewcommand{\lstlistingname}{Izpis}%
}
\makeatother
 %slovene captions for hyperref and algorithm2e
% Title Page
\begin{titlepage}    
    \title{\Huge Optimizacija zaznave trkov v 2D}
    \author{Klemen Javoršek}
    \date{Ljubljana, 2024/2025}
    \maketitle
    \renewcommand{\headrulewidth}{0cm}
    \fancyhf{}
    \fancyfoot[L]{Mentor: Klemen Bajec}
    \fancyfoot[R]{Gimnazija Vič}
    \thispagestyle{fancy}
\end{titlepage}
\selectlanguage{slovene}

% Table of Contents
\newpage    
\setcounter{page}{1}
\tableofcontents
\newpage


Implementirali smo fizikalni simulator v 2D prostoru, ki zaznava trke med omejujočimi volumni
v obliki kroga. Volumni se premikajo in imajo med sabo popolne elastične trke.
Osnovna zanka je sestavljena iz faze zaznave trkov med volumni, upoštevanja trkov in posledičnega
spreminjanja vektorjev objektov, premikanja objektov in optimizacije drevesa. Za učinkovitost
simulatorja je zelo pomembno, da je drevo čimbolj uravnoteženo, ter da je čim bolj plitko.
\section{Drevo objektov}
Naša implementacija fizikalnega simulatorja za pospeševanje zaznave trkov uporablja BSP drevo.
Drevo se razdeljuje po premicah, ki so vzporedne koordinatnima osema.

\lstset{
    numbers=left,
    numberstyle=\tiny,
    basicstyle=\ttfamily\small,
    keywordstyle=\color{violet},
    identifierstyle=\color{teal}
}

Drevo je sestavljeno iz vozlišč, ki so C strukture. Za preprostejšo uporabo je drevo povezano
navzdol in navzgor - vsako vozlišče ima kazalce na starševsko vozlišče in svoja dva otroka.
Razdeljevanje poteka s pomočjo dveh spremenljivk: smeri razdelitve (os $x$ ali os $y$) ter
poziciji razdelitve (primer: $1334.5$). Na tak način lahko funkcije, ki se spuščajo skozi drevo,
učinkovito določijo naslednje vozlišče, ki ga bodo obiskale\cite{klein_point_2004}. Objekti so v
drevesu shranjeni le v vozliščih, ki nimajo otrok, torej so na dnu svoje veje drevesa.
\begin{lstlisting}[float, caption={Struktura vozlišča}, label=vupdate_runner, language=C]
    struct treeNode {
        struct treeNode* left;
        struct treeNode* right;
        struct treeSplit split;
    //  vsebina strukture treeSplit:
    //      unsigned isx;
    //      double value;
        struct treeNode* up;
        object* buf; // NULL -> list
        uint64_t places; //bitmask: prazen = 1, zaseden = 0
        rect_llhh bindrect; //okvir, ki ga zaseda vozlisce
        unsigned level; // kako globoko v drevesu je vozlisce
        uint16_t flags;
        uint16_t optindex;
    };
\end{lstlisting}
Poleg povezav z ostalimi vozlišči in razdeljevanja ima vsako vozlišče tudi možnost kazalca (\lstinline|buf|) na
del spomina, kjer so shranjeni objekti. Ta kazalec ima tudi to praktično lastnost, da učinkovito
pokaže, ali je vozlišče na dnu svoje veje drevesa. Vključen je tudi pravokotnik, ki opisuje površino
vozlišča v simulaciji. Z njegovo pomočjo lahko ugotovimo, ali je treba objekt premestiti v drugo vozlišče.
Pomemben del vsakega vozlišča je bitna maska \lstinline|places|, ki označuje lokacijo vsakega objekta v kosu
pomnilnika, kjer so shranjeni. Na ta način lahko z eno bitno operacijo preverimo tudi, ali je v vozlišču
preveč objektov in ga je treba razdeliti. Z procesorskim ukazom \lstinline|popcount| lahko tudi hitro
preštejemo objekte.
Poleg tega je v strukturi tudi bitna maska \lstinline|flags|, ki omogoča označevanje vozlišča, če ima
kakšen poseben status.

\begin{figure}[]
    
    \vspace{0.2cm}
    \centering

    \tikz[tree layout, grow'=down, level distance=11mm, sibling distance=3mm,
          nodes={draw,fill=cyan!40,circle,inner sep=2pt, scale=0.6}
    ]
    \node{NULL}
    child {node {NULL}
      child {node {NULL}
      }
      child {node[fill=red!50]{0x7F...}
        child{node[draw, rectangle, fill=white]{[... , ...]}}
      }
    }
    child {node {NULL}
      child {node {NULL}}
      child {node {NULL}}
    };
    \caption{Drevo z vozliščem z objekti}
    \label{fig:drevo_z_buf}

\end{figure}


\subsection{Iskanje po drevesu}

Hitro najti objekt le s pomočjo njegovih koordinat je celoten razlog, zaradi katerega
sploh potrebujemo BSP drevo, zato je gladka in učinkovita implementacija iskalnih algoritmov
ključnega pomena. Obstajata dve vrsti iskanja:
Iskanje točke (objekta), ter iskanje območja (pravokotnika). Iskanje točke je pomembno za
vstavljanje in ponovno vstavljanje objektov v drevo, iskanje območja pa za zaznavo trkov.

\begin{lstlisting}[float, caption={Iskanje vozlišča, ki vsebuje objekt}, label=find_parent_node, language=C]
    treeNode* tree_findParentNode(objTree* tree, object* obj){
        treeNode* currentNode = tree->root;
        while(1) {
            if(currentNode->buf) return currentNode; // smo na dnu
            switch(currentNode->split.isx) {
                case 1: // x split
                    if(obj->x < currentNode->split.value){
                        currentNode = currentNode->left;
                    }
                    else {
                        currentNode = currentNode->right;
                    }
                    break;
                case 0: // y split
                    if(obj->y < currentNode->split.value){
                        currentNode = currentNode->left;
                    }
                    else {
                        currentNode = currentNode->right;
                    }
            }
        }
    }
\end{lstlisting}
Iskanje točke na izpisu \ref{find_parent_node} sicer ni rekurzivna, a pri
iskanju pravokotnika se srečamo z več potencialnimi zadetki.
\begin{samepage}
    Iskanje v drevesu poteka na preprost način (prikazan tudi na izpisu \ref{find_parent_node}):
    \begin{enumerate}
        \item Začnemo pri korenu drevesa.
        \item Preverimo, ali ima trenutno vozlišče kazalec na objekte.
        \item Če kazalec ima, potem vrnemo kazalec na vozlišče.
        \item Če kazalca nima:
        \begin{itemize}
            \item Preverimo, po kateri koordinati se razdeljuje vozlišče.
            \item Razdelitev primerjamo z relevantno koordinato objekta.
            \item Glede na primerjavo zamenjamo trenutno vozlišče na levega ali desnega otroka.
            \item Vrnemo se na korak 2.
        \end{itemize}    
    \end{enumerate}
\end{samepage}

\subsection{Grajenje drevesa}
Za grajenje dreves, ki organizirajo prostor, je veliko pristopov. Cilj vsakega pristopa je ponavadi čimbolj
uravnoteženo drevo. Večina jih je prilagojenih na to, da se objekte, ki jih organiziramo, bere v določenem
vrstnem redu. Ker je simulator, ki smo ga implementirali, namenjen le raziskovanju, lahko izberemo zelo 
preprost, a učinkovit način ustvarjanja uravnoteženega drevesa, vstavljanje naključnih objektov.

Ker se lahko zanesemo, da bo naključnost poskrbela za enakomerno, nesekvenčno vstavljanje objektov, se lahko
izognemo slabo zgrajenim, neuravnoteženim drevesom, kot je drevo na sliki \ref{fig:drevo_unbal}.
\begin{figure}
    \centering
    \includestandalone{snips_pics/unbal_tree}
    \caption{Neuravnoteženo drevo, če vstavljamo objekte, urejene po koordinatah.}
    \label{fig:drevo_unbal}
\end{figure}



Če hočemo objekt vstaviti v drevo, moramo najprej najti pravo vozlišče, nato preveriti,
ali je slučajno polno, in v tem primeru narediti dva otroka in jima razdeliti objekte.
Objekti so v svojem kosu pomnilnika razporejeni nepredvidljivo. Razlog za to je, da se nekateri
morali prestaviti v drugo vozlišče zaradi svoje pozicije zunaj pravokotnika, ki ga pokriva
njihovo prejšnje vozlišče.
\subsubsection{Zapis objektov v pomnilniku}
S tem se lahko učinkovito spopademo tako, da si zapišemo, katera mesta
v kosu pomnilnika z objekti so polna in katera so prazna. V ta namen uporabljamo \lstinline|uint64_t places|,
ki je del strukture \lstinline|treeNode|. S pomočjo takega 64-bitnega števila lahko z branjem in pisanjem
posameznih bitov spremljamo zasedenost kosa pomnilnika do največ 64 objektov, kar je več kot dovolj
za naše potrebe. Nekaj osnovnih operacij je opisanih v izpisu~\ref{bitmask_examples}:
\begin{lstlisting}[caption={Uporaba bitne maske za objekte}, label=bitmask_examples, language=C]
    // preveri, ali je na indeksu N objekt
    !(node->places & (1UL << N))
    // indeks prvega prostega mesta
    __builtin_ctzll__(node->places)
    // indeks prvega objekta
    __builtin_ctzll__(~node->places)
    // koliko je objektov
    __builtin_popcountl__(~node->places)
\end{lstlisting}

\newpage
\lstlistoflistings{}
\bibliographystyle{plain}
\bibliography{raziskovalna.bib}
% End Document
\end{document}
% =============================================================================================
% Sections and Content
\begin{comment}
    \section{Game plan}
    Pojavi in eksperimenti, ki jih bom opisoval v raziskovalni nalogi:
    \begin{itemize}
        \item Vpliv velikosti škatel na hitrost simulacije
        \item Vpliv gostote predmetov na hitrost
        \item Vpliv reciklaže škatel na hitrost
        \item Vpliv linearnega alokatorja vs malloc()
        \item 
    \end{itemize}
    Kako bodo videti poglavja?
    \begin{itemize}
        \item Teoretični uvod (kompleksnost, drevesa - k-d, BSP, ...)
        \begin{itemize}[label=$\diamond$, left=0.5cm]
            \item kompleksnost - zakaj ne bruteforce?
            \item vrste dreves\cite{klein_point_2004}
            \item zakaj k-d?
            \item opis podatkovnih struktur (slikice?)
        \end{itemize}
        \item Opis projekta
        \item 
    \end{itemize}
    
    Seveda moramo pomisliti na prvi.
    
    In drugi odstavek v tem besedilu.
    \end{comment}


\section{Uvod}
Uvod v zaznavo trkov (najdi dober članek, ki govori o tem?)
\section{Pregled področja}
    \begin{itemize}
        \item kompleksnost in bruteforce
        \item zato je podatkovna struktura pomembna
        \item obstajajo knjižnice - Kaj implementirajo?
        \item glavne podatkovne strukture, zakaj prav k-d drevo?
        \item razložim graf kompleksnosti ($n^2$ vs $n*\log(n)$)
    \end{itemize}
\section{Arhitektura implementacije}
    \subsection{Zaznava trkov}
    Kako delujejo moje funkcije?
    \subsection{Pomožne funkcije}
    Premikanje objektov naokrog in druge dejavnosti
    \subsection{Testno okolje}
    Kako merim podatke?
\section{Eksperimentalni del}
    Optimizacije (seznam in obrazložitev, eksp. podatki) TikZ za grafe
    \begin{itemize}
        \item Vpliv velikosti škatel na hitrost simulacije
        \item Vpliv gostote predmetov na hitrost
        \item Vpliv reciklaže praznih škatel na hitrost (WIP)
        \item Vpliv linearnega alokatorja vs malloc()
        \item Vpliv gravitacije (prostorskega zgoščevanja škatel) na hitrost
    \end{itemize}
\section{Zaključek}
Kaj sem naredil in pot naprej, primerjava z obstoječimi implementacijami
